\documentclass{article}

%% This template is pieces of templates provided by Chi-Kit Lam for
%% CS395T in S2016, bits taken from
%% https://github.com/nykh/latex-shorthands/blob/master/myshorthands.tex, and %% other configs scrapped together over time. 
\PassOptionsToPackage{svgnames}{xcolor}
\usepackage[margin=2cm]{geometry}
\usepackage{float,graphicx,tabularx}
\usepackage[labelformat=simple]{subcaption}
\usepackage{eqparbox}
\usepackage{bbm}
\usepackage{leftidx}
\usepackage{mathtools}
\usepackage{semantic}
\usepackage[version=4]{mhchem} % chemical formulae
\usepackage{hyperref}
\usepackage{parskip}  %Extra space on new paragraph, no indent
\usepackage{graphicx}
\usepackage{natbib}
\usepackage{listings}

\renewcommand\thesubfigure{(\alph{subfigure})}

%%% Section Separators

\newcommand{\starhsep}{
\begin{center}
  $\ast$~$\ast$~$\ast$
\end{center}}

\newcommand{\clubhsep}{
\begin{center}
  $\clubsuit$~$\clubsuit$~$\clubsuit$
\end{center}}

%% Paired delimiters, use with \delim{*expression*}
\DeclarePairedDelimiter\abkt{\langle}{\rangle}
\DeclarePairedDelimiter\cbkt{\lbrace}{\rbrace}
\DeclarePairedDelimiter\rbkt{\lparen}{\rparen}
\DeclarePairedDelimiter\sbkt{\lbrack}{\rbrack}
\DeclarePairedDelimiter\abs{\lvert}{\rvert}
\DeclarePairedDelimiter\norm{\lVert}{\rVert}
\DeclarePairedDelimiter\floor{\lfloor}{\rfloor}
\DeclarePairedDelimiter\ceiling{\lceil}{\rceil}
\DeclarePairedDelimiterX\set[2]{\lbrack}{\rbrack}{#1 \colon #2}

%% Operators

\newcommand{\clapover}[2]{\overbrace{#1}_{\mathclap{#2}}}
\newcommand{\clapunder}[2]{\underbrace{#1}_{\mathclap{#2}}}

\DeclareMathOperator*{\argmin}{arg\,min}
\DeclareMathOperator*{\argmax}{arg\,max}
\DeclareMathOperator{\supp}{supp}
\newcommand{\defeq}{\stackrel{\text{def}}=}

\DeclareMathOperator{\matrixtranspose}{T}
\newcommand{\T}{{\matrixtranspose}}
\DeclareMathOperator{\frobeniusnorm}{F}
\newcommand{\Fro}{{\frobeniusnorm}}
\DeclareMathOperator{\diag}{diag}

\DeclareMathOperator*{\E}{E}
\DeclareMathOperator*{\Var}{Var}
\DeclareMathOperator*{\Cov}{Cov}
\DeclareMathOperator*{\Corr}{Corr}
\DeclareMathOperator{\BinDist}{Bin}
\DeclareMathOperator{\GaussDist}{N}
\DeclareMathOperator{\PoisDist}{Pois}
\DeclareMathOperator{\UnifDist}{U}
\DeclareMathOperator{\erf}{erf} % error function.
\DeclareMathOperator{\sgn}{sgn}

\newcommand{\prox}{ \mathop{\mathrm{prox}} }
\newcommand{\enorm}[1]{\Vert #1 \Vert_2}

\DeclareMathOperator{\df}{d} % Differential form.
\DeclareMathOperator{\round}{\partial}

\DeclareMathOperator{\area}{area}
\DeclareMathOperator{\vol}{vol}

%% Variables. This may seem excessive but trust me it's quite nice

\newcommand{\eps}{\varepsilon}
\newcommand{\haat}{\widehat}

\newcommand{\bbA}{{\mathbb{A}}}
\newcommand{\bbB}{{\mathbb{B}}}
\newcommand{\bbC}{{\mathbb{C}}}
\newcommand{\bbD}{{\mathbb{D}}}
\newcommand{\bbE}{{\mathbb{E}}}
\newcommand{\bbF}{{\mathbb{F}}}
\newcommand{\bbG}{{\mathbb{G}}}
\newcommand{\bbH}{{\mathbb{H}}}
\newcommand{\bbI}{{\mathbb{I}}}
\newcommand{\bbJ}{{\mathbb{J}}}
\newcommand{\bbK}{{\mathbb{K}}}
\newcommand{\bbL}{{\mathbb{L}}}
\newcommand{\bbM}{{\mathbb{M}}}
\newcommand{\bbN}{{\mathbb{N}}}
\newcommand{\bbO}{{\mathbb{O}}}
\newcommand{\bbP}{{\mathbb{P}}}
\newcommand{\bbQ}{{\mathbb{Q}}}
\newcommand{\bbR}{{\mathbb{R}}}
\newcommand{\bbS}{{\mathbb{S}}}
\newcommand{\bbT}{{\mathbb{T}}}
\newcommand{\bbU}{{\mathbb{U}}}
\newcommand{\bbV}{{\mathbb{V}}}
\newcommand{\bbW}{{\mathbb{W}}}
\newcommand{\bbX}{{\mathbb{X}}}
\newcommand{\bbY}{{\mathbb{Y}}}
\newcommand{\bbZ}{{\mathbb{Z}}}
\newcommand{\bbone}{{\mathbbm{1}}}

\newcommand{\bfA}{{\mathbf{A}}}
\newcommand{\bfB}{{\mathbf{B}}}
\newcommand{\bfC}{{\mathbf{C}}}
\newcommand{\bfD}{{\mathbf{D}}}
\newcommand{\bfE}{{\mathbf{E}}}
\newcommand{\bfF}{{\mathbf{F}}}
\newcommand{\bfG}{{\mathbf{G}}}
\newcommand{\bfH}{{\mathbf{H}}}
\newcommand{\bfI}{{\mathbf{I}}}
\newcommand{\bfJ}{{\mathbf{J}}}
\newcommand{\bfK}{{\mathbf{K}}}
\newcommand{\bfL}{{\mathbf{L}}}
\newcommand{\bfM}{{\mathbf{M}}}
\newcommand{\bfN}{{\mathbf{N}}}
\newcommand{\bfO}{{\mathbf{O}}}
\newcommand{\bfP}{{\mathbf{P}}}
\newcommand{\bfQ}{{\mathbf{Q}}}
\newcommand{\bfR}{{\mathbf{R}}}
\newcommand{\bfS}{{\mathbf{S}}}
\newcommand{\bfT}{{\mathbf{T}}}
\newcommand{\bfU}{{\mathbf{U}}}
\newcommand{\bfV}{{\mathbf{V}}}
\newcommand{\bfW}{{\mathbf{W}}}
\newcommand{\bfX}{{\mathbf{X}}}
\newcommand{\bfY}{{\mathbf{Y}}}
\newcommand{\bfZ}{{\mathbf{Z}}}
\newcommand{\bfa}{{\mathbf{a}}}
\newcommand{\bfb}{{\mathbf{b}}}
\newcommand{\bfc}{{\mathbf{c}}}
\newcommand{\bfd}{{\mathbf{d}}}
\newcommand{\bfe}{{\mathbf{e}}}
\newcommand{\bff}{{\mathbf{f}}}
\newcommand{\bfg}{{\mathbf{g}}}
\newcommand{\bfh}{{\mathbf{h}}}
\newcommand{\bfi}{{\mathbf{i}}}
\newcommand{\bfj}{{\mathbf{j}}}
\newcommand{\bfk}{{\mathbf{k}}}
\newcommand{\bfl}{{\mathbf{l}}}
\newcommand{\bfm}{{\mathbf{m}}}
\newcommand{\bfn}{{\mathbf{n}}}
\newcommand{\bfo}{{\mathbf{o}}}
\newcommand{\bfp}{{\mathbf{p}}}
\newcommand{\bfq}{{\mathbf{q}}}
\newcommand{\bfr}{{\mathbf{r}}}
\newcommand{\bfs}{{\mathbf{s}}}
\newcommand{\bft}{{\mathbf{t}}}
\newcommand{\bfu}{{\mathbf{u}}}
\newcommand{\bfv}{{\mathbf{v}}}
\newcommand{\bfw}{{\mathbf{w}}}
\newcommand{\bfx}{{\mathbf{x}}}
\newcommand{\bfy}{{\mathbf{y}}}
\newcommand{\bfz}{{\mathbf{z}}}
\newcommand{\bfzero}{{\mathbf{0}}}
\newcommand{\bfone}{{\mathbf{1}}}
\newcommand{\bfGamma}{{\boldsymbol \Gamma}}
\newcommand{\bfDelta}{{\boldsymbol \Delta}}
\newcommand{\bfTheta}{{\boldsymbol \Theta}}
\newcommand{\bfLambda}{{\boldsymbol \Lambda}}
\newcommand{\bfXi}{{\boldsymbol \Xi}}
\newcommand{\bfPi}{{\boldsymbol \Pi}}
\newcommand{\bfSigma}{{\boldsymbol \Sigma}}
\newcommand{\bfUpsilon}{{\boldsymbol \Upsilon}}
\newcommand{\bfPhi}{{\boldsymbol \Phi}}
\newcommand{\bfPsi}{{\boldsymbol \Psi}}
\newcommand{\bfOmega}{{\boldsymbol \Omega}}
\newcommand{\bfalpha}{{\boldsymbol \alpha}}
\newcommand{\bfbeta}{{\boldsymbol \beta}}
\newcommand{\bfgamma}{{\boldsymbol \gamma}}
\newcommand{\bfdelta}{{\boldsymbol \delta}}
\newcommand{\bfeps}{{\boldsymbol \eps}}
\newcommand{\bfzeta}{{\boldsymbol \zeta}}
\newcommand{\bfeta}{{\boldsymbol \eta}}
\newcommand{\bftheta}{{\boldsymbol \theta}}
\newcommand{\bfiota}{{\boldsymbol \iota}}
\newcommand{\bfkappa}{{\boldsymbol \kappa}}
\newcommand{\bflambda}{{\boldsymbol \lambda}}
\newcommand{\bfmu}{{\boldsymbol \mu}}
\newcommand{\bfnu}{{\boldsymbol \nu}}
\newcommand{\bfxi}{{\boldsymbol \xi}}
\newcommand{\bfpi}{{\boldsymbol \pi}}
\newcommand{\bfrho}{{\boldsymbol \rho}}
\newcommand{\bfsigma}{{\boldsymbol \sigma}}
\newcommand{\bftau}{{\boldsymbol \tau}}
\newcommand{\bfupsilon}{{\boldsymbol \upsilon}}
\newcommand{\bfphi}{{\boldsymbol \phi}}
\newcommand{\bfchi}{{\boldsymbol \chi}}
\newcommand{\bfpsi}{{\boldsymbol \psi}}
\newcommand{\bfomega}{{\boldsymbol \omega}}

\newcommand{\calA}{{\mathcal{A}}}
\newcommand{\calB}{{\mathcal{B}}}
\newcommand{\calC}{{\mathcal{C}}}
\newcommand{\calD}{{\mathcal{D}}}
\newcommand{\calE}{{\mathcal{E}}}
\newcommand{\calF}{{\mathcal{F}}}
\newcommand{\calG}{{\mathcal{G}}}
\newcommand{\calH}{{\mathcal{H}}}
\newcommand{\calI}{{\mathcal{I}}}
\newcommand{\calJ}{{\mathcal{J}}}
\newcommand{\calK}{{\mathcal{K}}}
\newcommand{\calL}{{\mathcal{L}}}
\newcommand{\calM}{{\mathcal{M}}}
\newcommand{\calN}{{\mathcal{N}}}
\newcommand{\calO}{{\mathcal{O}}}
\newcommand{\calP}{{\mathcal{P}}}
\newcommand{\calQ}{{\mathcal{Q}}}
\newcommand{\calR}{{\mathcal{R}}}
\newcommand{\calS}{{\mathcal{S}}}
\newcommand{\calT}{{\mathcal{T}}}
\newcommand{\calU}{{\mathcal{U}}}
\newcommand{\calV}{{\mathcal{V}}}
\newcommand{\calW}{{\mathcal{W}}}
\newcommand{\calX}{{\mathcal{X}}}
\newcommand{\calY}{{\mathcal{Y}}}
\newcommand{\calZ}{{\mathcal{Z}}}

%% Special Group Aliases

\DeclareMathOperator{\groupGF}{GF}
\DeclareMathOperator{\groupGL}{GL}
\DeclareMathOperator{\groupSL}{SL}
\DeclareMathOperator{\groupO}{O}
\DeclareMathOperator{\groupSO}{SO}
\DeclareMathOperator{\groupU}{U}
\DeclareMathOperator{\groupSU}{SU}
\DeclareMathOperator{\groupE}{E}
\DeclareMathOperator{\groupSE}{SE}

\mathlig{---}{\text{---}}

%% Some sane listings defaults

\usepackage{color}

\definecolor{mygreen}{rgb}{0,0.6,0}
\definecolor{mygray}{rgb}{0.5,0.5,0.5}
\definecolor{mymauve}{rgb}{0.58,0,0.82}

\lstset{ %
  backgroundcolor=\color{white},   % choose the background color; you must add \usepackage{color} or \usepackage{xcolor}
  basicstyle=\footnotesize,        % the size of the fonts that are used for the code
  breakatwhitespace=false,         % sets if automatic breaks should only happen at whitespace
  breaklines=true,                 % sets automatic line breaking
  captionpos=b,                    % sets the caption-position to bottom
  commentstyle=\color{mygreen},    % comment style
  deletekeywords={...},            % if you want to delete keywords from the given language
  escapeinside={\%*}{*)},          % if you want to add LaTeX within your code
  extendedchars=true,              % lets you use non-ASCII characters; for 8-bits encodings only, does not work with UTF-8
  frame=single,                    % adds a frame around the code
  keepspaces=true,                 % keeps spaces in text, useful for keeping indentation of code (possibly needs columns=flexible)
  keywordstyle=\color{blue},       % keyword style
  language=Octave,                 % the language of the code
  otherkeywords={*,...},           % if you want to add more keywords to the set
  numbers=left,                    % where to put the line-numbers; possible values are (none, left, right)
  numbersep=5pt,                   % how far the line-numbers are from the code
  numberstyle=\tiny\color{mygray}, % the style that is used for the line-numbers
  rulecolor=\color{black},         % if not set, the frame-color may be changed on line-breaks within not-black text (e.g. comments (green here))
  showspaces=false,                % show spaces everywhere adding particular underscores; it overrides 'showstringspaces'
  showstringspaces=false,          % underline spaces within strings only
  showtabs=false,                  % show tabs within strings adding particular underscores
  stepnumber=2,                    % the step between two line-numbers. If it's 1, each line will be numbered
  stringstyle=\color{mymauve},     % string literal style
  tabsize=2,                     % sets default tabsize to 2 spaces
  title=\lstname                   % show the filename of files included with \lstinputlisting; also try caption instead of title
}


%%% style.sty ends%%%

\title{SDS 385 Exercise Set}
\author{Kevin Song}
\date{\today}

\begin{document}

\maketitle

\section*{The Proximal Operator}

\subsection*{(A)}

The proximal operator for the linear approximation of $f$ at $x_0$ can be
written and reduced as follows:

\begin{align*}
  \prox\limits_\gamma \hat{f}(x ; x_0)
  &= \prox\limits_\gamma \left[ f(x_0) + (x - x_0)^T\nabla f(x_0) \right]\\
  &= \argmin\limits_z \left[ f(x_0) + (z - x_0)^T \nabla f(x_0) +
    \frac{1}{2\gamma} \norm{z - x}_2^2 \right]\\
  &= \argmin\limits_z \left[ (z - x_0)^T \nabla f(x_0) +
    \frac{1}{2\gamma} \norm{z - x}_2^2 \right]\\
  &= \argmin\limits_z \left[ z^T \nabla f(x_0) - x_0^T \nabla f(x_0) +
    \frac{1}{2\gamma} \left( z^Tz - 2 z^T + x^Tx \right) \right]\\
  &= \argmin\limits_z \left[ z^T \nabla f(x_0) +
    \frac{1}{2\gamma} \left( z^Tz - 2 z^Tx \right) \right]\\
\end{align*}

where, in the last step, I have thrown away any terms
that are constant with respect to $z$. Since we are taking the argmin over $z$,
these terms are unimportant---alternatively, since we need to take the gradient
with respect to $z$ to solve this minimization, these terms will be zero
anyways.

To solve for this argmin, we take the gradient of this expression w.r.t z and
set the result equal to zero:

\begin{align*}
  \nabla_z& \left[ z^T \nabla f(x_0) +
            \frac{1}{2\gamma} \left( z^Tz - 2 z^Tx \right) \right]\\
          &= \nabla f(x_0) + \frac{1}{\gamma}(z - x) = 0\\
          &\implies z = x - \gamma \nabla f(x_0)
\end{align*}

This shows that the solution to the proximal operator of the linear
approximation of the function is the gradient descent step.

\subsection*{(B)}

Consider a log-likelihood of the form $\ell(x) = \frac{1}{2} x^TPx - q^Tx + r$.
The proximal operator of this function, with parameter $\frac{1}{\gamma}$ is

\begin{align*}
  \prox\limits_{\frac{1}{\gamma}} \ell(x)
  &= \prox\limits_{\frac{1}{\gamma}} \frac{1}{2} x^TPx - q^Tx + r\\
  &= \argmin_z \frac{1}{2} z^TPz - q^Tz + r + \frac{\gamma}{2} (z - x)^T(z - x)\\
  &= \argmin_z \frac{1}{2} \left( z^TPz + \gamma z^T I z \right)
    - (q^Tz + \gamma x^Tz)  + r + x^Tx \\
  &= \argmin_z \frac{1}{2} z^T(P + \gamma I)z - (q + \gamma x)^T z + r + x^Tx
\end{align*}

We know that the minimum of the quadratic form $\frac{1}{2} z^TAz + b^Tz + c$ is
given by the solution to $Az - b = 0$ or $z = A^{-1}b$, so the minimum to this
likelihood is

\[
  \prox\limits_{\frac{1}{\gamma}} \ell(x) = \left( P + \gamma I \right)^{-1} (\gamma
  x + q)
\]


In part B, the likelihood of such a sample is (PDF from Wikipedia because I'm lazy)

\[
  \calL (y_1, \dots, y_n ; x) = \frac{1}{\left( \sqrt{2\pi}\right)^k
    \sqrt{\det{\Omega^{-1}}}} \prod_{i=1}^n \exp \left( -\frac{1}{2}(y_i - Ax)^T
  \Omega(y_i - Ax)\right)
\]

Since our ultimate goal is to minimize the likelihood, we can immediately drop
the constant terms. Taking the log of both sides, we find that

\begin{align*}
  \log \calL (y_1, \dots, y_n ; x)
  &= \log \frac{1}{\sqrt{\det{\Omega^{-1}}}} \prod_{i=1}^n \exp \left( -\frac{1}{2}(y_i - Ax)^T
    \Omega(y_i - Ax)\right)\\
  &= \log \frac{1}{\sqrt{\det{\Omega^{-1}}}} + \sum_{i=1}^n \log \exp \left( -\frac{1}{2}(y_i - Ax)^T
    \Omega(y_i - Ax)\right)\\
  &=-\left( \log \sqrt{\det{\Omega^{-1}}} + \frac{1}{2} \sum_{i=1}^n(y_i - Ax)^T
    \Omega(y_i - Ax)\right) 
\end{align*}

This shows us that

\[
 - \log \calL =  \frac{1}{2} \log \det{\Omega^{-1}} + \frac{1}{2} \sum_{i=1}^n(y_i - Ax)^T
    \Omega(y_i - Ax)
\]

This can be fit into the quadratic form by choose $P = \Omega$, $q = \vec{0}$,
and $r = \frac{1}{2} \log \det{\Omega^{-1}}$.

\subsection*{(C)}

\[
\prox_\gamma \phi(x) = \argmin_z \tau \norm{z}_1 + \frac{1}{2\gamma} \norm{z-
  x}_2^2 = \argmin_z \tau \sum_i |z_i| + \frac{1}{2\gamma} \sum_i (z_i - x_i)^2
\]

Since the $z_i$ are independent of each other, by minimizing each element of the
summation, we minimize the overall sum (this would not be true if $z$ were
constrained in some form).

From Exercise 5, we know that the solution to

\[
 \argmin_{z_i} \tau |z_i| + \frac{1}{2\gamma} (z_i - x_i)^2
\]

is given by

\[
  \mathrm{sign}(x_i)(|x_i| - \gamma\tau)_{+}
\]


\section*{Proximal Gradient Method}

\subsection*{(A)}

To prove this, we show that the provided form produces the correct solution:

\begin{alignat*}{2}
  \prox_\gamma \phi(u)
  &= \argmin_z \phi(z) - \frac{1}{2\gamma} \norm{z - x_0 + \gamma \nabla
    \ell(x_0)}_2^2\\
  = \argmin_z \phi(z) &- \frac{1}{2\gamma} [ z^Tz - z^Tx_0 +
    z^T \gamma\nabla\ell(x_0)
    x_0^Tz + x_0^Tx_0 - x_0^T\gamma \nabla\ell(x_0)\\
    &+  \gamma \nabla \ell(x_0)^Tz + \gamma\nabla\ell(x_0)^Tx_0 +
    \gamma^2 \nabla\ell(x_0)^T\nabla\ell(x_0) ]
\end{alignat*}

where the second step involves explicitly multiplying out the norm. We can now
rearrange this mess and remove some constant terms to reveal that

\begin{align*}
    &\argmin_z \left[ \phi(z) - \frac{1}{2\gamma} \left(  [z^Tz - 2z^Tx_0 + x_0^Tx_0]  + 2\gamma z^T \nabla \ell(x_0)
                \right)  \right]\\
    &= \argmin_z \left[ \phi(z) - \frac{1}{2\gamma} \norm{z-x}_2^2  + z^T \nabla \ell(x_0) \right]
\end{align*}

which is the exact same minimization problem as minimizing $\widetilde{f}$ (up
to alpha equivalence, replacing the name $z$ with $x$).

\subsection*{(B)}

\begin{lstlisting}[mathescape=true]
  function calc_gradient(X,y,$\beta$){
     // We already know how to do this, omitted for brevity
  }

  function calc_gamma(){
    return 0.42 //Not going to worry about gamma yet. Just use a constant
  }

  function solve_prox($x$, $\gamma$){
    for each $x_i${
      $z_i$ = sign($x_i$) * max((abs($x_i$) $-$ $\gamma$), 0)
    }
  }

  function proximal_gradient(X,y,$\beta$){
    repeat until convergence{
       grad = calc_gradient(X,y,$\beta$)
       $\gamma$ = calc_gamma()
       u = $\beta - \gamma$*grad
       $\beta$ = solve_prox(u, $\gamma$)
    }
  }
\end{lstlisting}

The big costs in this function are calculating the gradient. Smaller (linear) costs are
associated with updating $u$ and solving the proximal operator.

\end{document}
